\chapter{Einleitung}\label{ch:einleitung}

Kubernetes ist eine moderne Technologie, die skalierbare Applikationen ermöglicht.
Sie beruht auf dem Prinzip, mehrere Rechner miteinander zu vernetzen und ihre gesamten Ressourcen effizient zu nutzen.
In Produktivumgebungen werden dazu leistungsstarke Maschinen oder Cloud-Instanzen eingesetzt.
% TODO finanz nicht so wichtig, vorteile von lokal hervorheben, warum überhaupt cluster? fork computing, mobile edge computing
Im Entwicklungsbetrieb kann es jedoch von Vorteil sein, aus finanziellen Gründen auf die Leistungsfähigkeit und die gute Anbindung eines Rechenzentrums zu verzichten und stattdessen auf lokal betriebene Systeme zu setzen.
% TODO raspi verlinken
Eine besonders günstige Option stellen hierbei Einplatinencomputer wie der Raspberry Pi dar.

Es sind viele Schritte nötig, um einen Kubernetes-Cluster einzurichten und je mehr Nodes eingerichtet werden sollen, umso häufiger müssen die immer gleichen Schritte durchgeführt werden.
Mithilfe des Automatisierungs-Werkzeugs Ansible können diese Aufwände automatisiert und somit vereinfacht und beschleunigt werden.
Nachdem mit wenigen Handgriffen das Standardbetriebssystem Raspbian installiert wurde, werden alle weiteren Schritte von Ansible-Playbooks automatisch erledigt.

In dieser Seminararbeit werden zunächst die verwendeten Technologien, Kubernetes und Ansible, kurz vorgestellt (Kapitel~\ref{ch:technologien}).
Anschließend wird die Vorgehensweise zum Aufsetzen eines Clusters mithilfe der Playbooks übersichtlich zusammengefasst (Kapitel~\ref{ch:anwendung}).
Danach erfolgt eine ausführliche Erläuterung der Funktionsweise der Playbooks (Kapitel~\ref{ch:umsetzung}).
Zuletzt werden mögliche Alternativen zu den eingesetzten Technologien vorgestellt (Kapitel~\ref{ch:alternativen}) und ein Ausblick auf mögliche Weiterentwicklungen gegeben (Kapitel~\ref{ch:zusammenfassung}).