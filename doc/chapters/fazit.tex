\chapter{Fazit}\label{ch:fazit}

Die für dieses Projekt entwickelten Playbooks ermöglichen es, mit wenigen Handgriffen in kurzer Zeit einen Kubernetes-Cluster einzurichten.
Die benötigte Zeit für die gesamte Einrichtung beträgt circa 10 Minuten pro Raspberry Pi plus etwa 30 Minuten für die Einrichtung des Clusters.
Der überwiegende Teil dieser Zeit ist Wartezeit.
Der Einrichtungsvorgang läuft fast komplett selbstständig ab.
Eine manuelle Einrichtung des Clusters nimmt mehrere Stunden in Anspruch.

Der Ersparnis von wenigen Stunden bei der Anwendung der Ansible-Playbooks muss aber auch der Aufwand gegenübergestellt werden, der nötig ist, um diese Playbooks zu entwickeln.
In meinem Fall waren dafür weit über zehn Stunden nötig.
Für die Einrichtung von wenigen oder nur einem Cluster lohnt es sich nicht, die Schritte zu automatisieren.
Erst mit der häufigeren Nutzung der Playbooks, gegebenenfalls durch mehrere Nutzer, amortisieren sich die Aufwände.

\section{Ausblick}\label{sec:ausblick}

An verschiedenen Stellen bietet das Projekt Verbesserungspotenziale.
Einige davon sollen nun erläutert werden.

\subsection{Raspbian mit Ansible installieren}

Bislang wird zu Beginn der Einrichtung mit Balena Etcher ein separates Werkzeug eingesetzt, um Raspbian auf der Speicherkarte zu installieren (siehe Abschnitt~\ref{sec:raspbian-installieren}).
Erst danach wird Ansible zur weiteren Einrichtung verwendet.

Es wäre auch möglich, im Ansible-Playbook \texttt{local-""raspbian"".yaml} mit einem Systemaufruf von \texttt{dd} das Raspbian-Abbild auf die Speicherkarte zu schreiben.
Der vorhergehende Schritt mit Balena Etcher könnte dadurch ersetzt werden.
Ein Nachteil wäre, dass der Anwender die Datenträgerbezeichnung (zum Beispiel \texttt{/dev/sdb}) selbst angeben müsste.

\subsection{Bestehenden Kubernetes-Cluster erhalten}

Das Ansible-Playbook \texttt{kubernetes"".yaml} richtet stets einen neuen Kubernetes-Cluster ein.
Falls schon vor Ausführung des Playbooks ein Cluster existiert, wird dieser gelöscht.
In manchen Situationen kann dieses Verhalten unerwünscht sein, zum Beispiel, wenn der Cluster um zusätzliche Nodes erweitert werden soll.
Besser wäre es, vor dem Initialisieren des Clusters zu überprüfen, ob bereits ein Cluster existiert und gegebenenfalls die Neueinrichtung zu überspringen.

\subsection{Globale Ansible-Variablen}

Beim Beitritt der Worker-Nodes zum Cluster (siehe Abschnitt~\ref{subsec:kubernetes-nodes-verbinden}) wird der vom Master-Node generierte Join-Befehl in eine Skriptdatei geschrieben, die auf die Worker-Nodes kopiert und dort ausgeführt wird.
Diese Vorgehensweise ist nicht optimal.

Effizienter wäre es, den Befehl in einer Ansible-Variable zu registrieren und ihn dann direkt auf den Worker-Nodes auszuführen.
Die Herausforderung dabei ist, dass Variablen in Ansible grundsätzlich nicht Host-übergreifend verwendet werden können.
Ansible-Facts sollten es ermöglichen, diesen Umstand zu umgehen.
Dies hat aber zum Zeitpunkt der Umsetzung dieses Projekts nicht funktioniert.
