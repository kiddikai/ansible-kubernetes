\chapter{Umsetzung}
Schöne Flowcharts?

Um das Ziel der maximalen Automatisierung zu erreichen, werden so viele Arbeitsschritte wie möglich von Ansible-Playbooks übernommen.
Die zwei relevanten Playbooks übernehmen unterschiedliche Aufgaben und unterscheiden sich grundsätzlich in ihrer Funktionsweise.
Bevor mit \texttt{kubernetes.yaml} die eigentliche Einrichtung des Clusters per gleichzeitigem Zugriff auf die Nodes via SSH vorgenommen werden kann, müssen die Systeme mit \texttt{local-raspbian} zunächst auf den Headless-Betrieb vorbereitet werden.
Dies geschieht ausschließlich über lokale Aktionen mit direktem Zugriff auf das Dateisystem der SD-Karten.

\section{Raspbian einrichten (\texttt{local-raspbian})}

Ein frisch geflashtes Raspbian ist zum vorgesehenen Einsatz über WiFi nicht geeignet, da die Zugangsdaten (SSID und WPA-Key) nicht bekannt sind.
Es muss also eine direkte Konfiguration des Systems erfolgen, bevor ein Node ans Netz gehen und sich regulär per Ansible konfigurieren lassen kann.
Die SD-Karte muss ohnehin einmal an einem separaten PC mit einem System-Image beschrieben werden, daher bietet sich dieser Moment an, um weitere Konfigurationen vorzunehmen.

Primär übernimmt das Playbook \texttt{local-raspbian} also die Einrichtung der WiFi-Zugangsdaten und einer statischen IP-Adresse.
Sämtliche Tasks in diesem Playbook werden nur auf dem Ansible-Host ausgeführt, nicht auf Remote-Hosts; und alle Aktionen erfolgen mittels direktem Eingriff ins Dateisystem, anstatt die Shell eines laufenden Systems anzusprechen.
Dadurch stehen die meisten Ansible-Module nicht zur Verfügung.

\subsection{Partitionen mounten}

Ein Raspbian-System besteht aus zwei Partitionen: eine Hauptpartition (rootfs), deren Struktur dem eines gewöhnlichen Linux-Systems entspricht und eine Boot-Partition, die dem leichten Zugriff auf häufig benötigte Einstellungen dient, ohne den Raspberry Pi booten zu müssen.
Beide Partitionen tragen eine eindeutige UUID, über die sie im Playbook zunächst mithilfe des Ansible-Moduls \texttt{mount} gemountet werden.

\subsection{Statische IP-Adresse setzen}

Standardmäßig werden IP-Adressen dynamisch über DHCP bezogen.
Da das Ansprechen über Hostnamen vom Netzwerk abhängt und nicht immer zuverlässig funktioniert, wird stattdessen eine statische IP-Adresse vergeben.
Dafür wird das Ansible-Inventory ausgelesen und auf die höchste bisher vergebene IP-Adresse 1 addiert und die resultierende Adresse sowie die Adresse des Routers und die Subnetz-Maske in die Datei \texttt{/etc/""dhcpcd.conf} geschrieben.

Zusätzlich wird in Abhängigkeit von der IP-Adresse ein sprechender Name als Hostname gewählt und dieser in den Dateien \texttt{/etc/""hostname} und \texttt{/etc/hosts} eingetragen.

\subsection{SSH-Daemon aktivieren}

Der SSH-Dienst von Raspbian ist von Haus aus deaktiviert, wird aber von Ansible benötigt.
Da SSH von vielen Anwendern verwendet wird, lässt sich der Daemon in Raspbian unkompliziert durch das Anlegen einer Datei \texttt{ssh} auf der Boot-Partition bewältigen.

\subsection{WiFi konfigurieren}

- /etc/wpa_supplicant/wpa_supplicant.conf
- rfkill

\subsection{Swapfile deaktivieren}



\subsection{kernel command line parameters (cgroup memory)}



\subsection{SSH-Keys hinterlegen}



