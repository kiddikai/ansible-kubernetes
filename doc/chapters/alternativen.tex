\chapter{Alternativen}\label{ch:alternativen}
\section{Alternativen zu Ansible}\label{sec:ansible-alternativen}

Ansible hat mehrere Konkurrenzprodukte, zum Beispiel Puppet und Chef.

Alle sind etabliert und eignen sich zur Automatisierung von Konfigurationen über Netzwerkverbindungen.
Ein wesentlicher Unterschied besteht in der grundlegenden Funktionsweise.
Während Ansible genau dann arbeitet, wenn ein Playbook auf dem Steuerungsrechner ausgeführt wird (Push-Prinzip), setzen Puppet und Chef Agenten-Software auf den Remote-Hosts voraus, die selbst den Zeitpunkt der Konfiguration bestimmen Pull-Prinzip).
Dafür muss dauerhaft ein Puppet- beziehunsgsweise Chef-Server bereitstehen, um zum gegebenen Zeitpunkt die nötigen Konfigurationen bereitzustellen.
Puppet und Chef eignen sich dadurch eher für Anwendungsfälle, in denen Zeit keine wichtige Rolle spielt oder nicht alle Remote-Hosts gleichzeitig verfügbar sind, zum Beispiel Büro-Rechner, die für begrenzte Dauer und zu unterschiedlichen Zeiten verwendet werden.

In diesem Projekt laufen alle Remote-Hosts gleichzeitig und eine unverzügliche Ausführung der Konfiguration auf Knopfdruck ist erwünscht.
Das macht Ansible zu einer guten Wahl.

\section{Alternativen zu Kubernetes}\label{sec:kubernetes-alternativen}
Docker Swarm?

Fleet?

Ubuntu?