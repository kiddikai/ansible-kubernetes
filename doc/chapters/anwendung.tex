\chapter{Anwendung}

%TODO Kurze Einleitung

\section{Vorbereitung}
Zum erfolgreichen Ausführen dieser Anleitung müssen folgende Voraussetzungen erfüllt sein:

\begin{itemize}
    \item \textbf{Raspberry Pis} in beliebiger Anzahl und ebenso viele Speicherkarten und Netzteile stehen bereit.
    \item \textbf{Ein Linux-Rechner} steht bereit, um die Speicherkarten zu flashen und die Ansible-Playbooks auszuführen. Dafür sind Balena Etcher\footnote{\url{https://www.balena.io/etcher/}} und Ansible\footnote{z. B. \texttt{apt install ansible}} installiert, das Git-Repo zu diesem Projekt mit den Verzeichnissen \texttt{inventory} und \texttt{playbook} ist ausgecheckt und ein Image von Raspbian Lite\footnote{\url{https://downloads.raspberrypi.org/raspbian_lite/images/raspbian_lite-2020-02-14/2020-02-13-raspbian-buster-lite.zip}} ist heruntergeladen.
    \item \textbf{Ein Terminal} mit \texttt{playbook} als Current Working Directory ist geöffnet.
    \item \textbf{Ein WiFi-Access Point} mit Internetzugriff ist in Betrieb. Seine Einstellungen (IP, SSID, WPA2-Key) entsprechen den Angaben in den Dateien \texttt{inventory/""group\_vars/""all.yaml} und \texttt{playbook/""local-""raspbian.yaml}. Der zuvor erwähnte Rechner ist mit dem Access Point verbunden.
    \item \textbf{Die Inventory-Datei} \texttt{inventory/""k8s-cluster.yaml} bildet den dereitigen Cluster ab -- enthält also keine Einträge unter \texttt{hosts}, falls ein neuer Cluster eingerichtet werden soll:

        \begin{lstlisting}
nodes:
  hosts:
        \end{lstlisting}
        \caption{Leere Inventory-Datei}
    \item Die benötigte Zeit für die gesamte Einrichtung beträgt ca. 10 Minuten pro Raspberry Pi plus ca. 30 Minuten für den gesamten Cluster.
\end{itemize}

\section{Raspbian installieren}
Die Schritte in diesem Abschnitt müssen für jeden Raspberry Pi einzeln durchgeführt werden.

\begin{enumerate}
    \item Speicherkarte in den Linux-Rechner einlegen.
    \item Raspbian-Image mit Balena Etcher auf die Speicherkarte flashen.
    \item Raspbian-Playbook ausführen:
        \begin{lstlisting}
ansible-playbook -i ../inventory/k8s-cluster.yaml local-raspbian.yaml
        \end{lstlisting}
        \caption{CLI-Befehl zur Ausführung des Raspbian-Playbooks}
    \item Speicherkarte in den Raspberry Pi einsetzen und booten.
\end{enumerate}

\section{Cluster aufsetzen}

Wenn alle Raspberry Pis gestartet sind, wird die Installation mit dem Kubernetes-Playbook fortgesetzt:
\begin{lstlisting}
ansible-playbook -i ../inventory/k8s-cluster.yaml kubernetes.yaml
\end{lstlisting}
\caption{CLI-Befehl zur Ausführung des Raspbian-Playbooks}

Der Kubernetes-Cluster ist anschließend einsatzbereit.

\section{Weitere Nodes hinzufügen}

Sollen zu einem fertigen Cluster weitere Nodes hinzugefügt werden, kann auch dafür diese Anleitung verwendet werden.
Die Inventory-Datei darf dann nicht leer sein, sondern muss die bereits vorhandenen Nodes enthalten.
