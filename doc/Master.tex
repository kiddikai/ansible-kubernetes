%% LyX 2.3.3 created this file.  For more info, see http://www.lyx.org/.
%% Do not edit unless you really know what you are doing.
\documentclass[oneside,english,a4paper,11pt,final,DIV13,BCOR20mm,openbib,openright,onecolumn,pagesize,pointlessnumbers,bibtotoc,liststotoc,idxtotoc,bigheadings,titlepage,tocindent,intoc,pdftex,utf8]{scrbook}
\usepackage[T1]{fontenc}
\usepackage[utf8]{inputenc}
\setcounter{secnumdepth}{3}
\setcounter{tocdepth}{3}

\makeatletter
%%%%%%%%%%%%%%%%%%%%%%%%%%%%%% User specified LaTeX commands.
\NeedsTeXFormat{LaTeX2e}
%\documentclass[a4paper,11pt,final,DIV13,BCOR20mm,openbib,twoside, openright,onecolumn,pagesize,pointlessnumbers,bibtotoc,liststotoc,idxtotoc,
%	bigheadings,titlepage,tocindent,intoc,pdftex]{scrbook}
\usepackage{emptypage}
\usepackage[ngerman]{babel}
%\usepackage[utf8]{inputenc}
%\usepackage[ansinew]{inputenc}
%\usepackage{blindtext}
%\usepackage[T1]{fontenc}
%\usepackage{bibgerm}
%\usepackage{lmodern}
%\usepackage{url}
%\usepackage{microtype}
%\usepackage{tabularx}
%\usepackage{xcolor}
%\usepackage[pdftex]{graphicx}
%\usepackage{amsmath}

%\usepackage{amsthm}
%\usepackage{amssymb}
%\usepackage{pdfpages}
%%\usepackage{subfig}
%\usepackage{subfigure}
%\usepackage{rotating}
%\usepackage{nextpage}
%\usepackage[numbers,sort&compress]{natbib}
%\usepackage{float}
%\usepackage{algorithm}
%\usepackage{algorithmicx}
\makeatletter
%\renewcommand*{\ALG@name}{Algorithmus}
\makeatother
\usepackage{algpseudocode}
\algtext*{EndIf}
\usepackage{lscape}
\usepackage[absolute]{textpos}
\setlength{\TPHorizModule}{1mm}
\setlength{\TPVertModule}{\TPHorizModule}
\textblockorigin{0mm}{0mm}
\usepackage{parskip}
\begingroup
\expandafter\expandafter\expandafter\endgroup
\expandafter\ifx\csname chapterformat\endcsname\relax\else
    \renewcommand*{\chapterformat}{
        \llap{
            \chapappifchapterprefix{\ }\thechapter\autodot\enskip
        }
    }
\fi
\renewcommand*{\othersectionlevelsformat}[1]{
    \llap{
        \csname the#1\endcsname\autodot\enskip
    }
}
\setlength{\unitlength}{1mm}
\newlength{\grafikdim}
\usepackage{listings}
\usepackage{xcolor}
\definecolor{backgroundcolor}{rgb}{0.95,0.95,0.95}
\lstdefinestyle{mystyle}{
    backgroundcolor=\color{backgroundcolor},
    commentstyle=\color{codegreen},
    keywordstyle=\color{magenta},
    numberstyle=\tiny\color{codegray},
    stringstyle=\color{codepurple},
    basicstyle=\ttfamily\footnotesize,
    breakatwhitespace=false,
    breaklines=true,
    captionpos=b,
    keepspaces=true,
    numbers=none,
    %numbersep=5pt,
    showspaces=false,
    showstringspaces=false,
    showtabs=false,
    tabsize=2
}

\lstset{style=mystyle}
\usepackage{color}
\renewcommand{\ttdefault}{pcr}
\renewcommand{\lstlistingname}{Codeausschnitt}
%JAVA
\definecolor{darkblue}{rgb}{0,0,.6}
\definecolor{darkred}{rgb}{.6,0,0}
\definecolor{forestgreen}{RGB}{34,139,34}
\definecolor{lgreen}{RGB}{50,201,50}
\definecolor{red}{rgb}{.98,0,0}
\definecolor{lbrown}{rgb}{0.54,0.27,0.01}
\definecolor{purple}{rgb}{0.5,0,0.33}
%XML
\definecolor{forestgreen}{RGB}{34,139,34}
\definecolor{orangered}{RGB}{239,134,64}
\definecolor{gray}{rgb}{0.4,0.4,0.4}
\lstdefinestyle{JAVA}{
  language=JAVA,
  captionpos=b, % Beschriftung ist unterhalb
  basicstyle=\ttfamily\small, % Schriftart&Größe
  commentstyle=\itshape\color{forestgreen},
  keywordstyle=\bfseries\color{purple},
  stringstyle=\color{blue},
  showspaces=false,
  showtabs=false,
  columns=fixed,
  frame=none,
  breaklines=true,
  showstringspaces=false,
  frame=tb,
	moredelim=[is][\itshape]{__}{__},
	moredelim=[is][\textcolor{lbrown}]{~~}{~~},
	moredelim=[is][\textcolor{blue}]{++}{++},
	moredelim=[is][\itshape\bfseries\textcolor{darkblue}]{**}{**}
}
\lstdefinestyle{JAVA_LN}{
  language=JAVA,
  captionpos=b, % Beschriftung ist unterhalb
  basicstyle=\ttfamily\small, % Schriftart&Größe
  commentstyle=\itshape\color{forestgreen},
  keywordstyle=\bfseries\color{purple},
  stringstyle=\color{blue},
  showspaces=false,
  showtabs=false,
  columns=fixed,
  frame=none,
  numbers=left, % Zeilennummern links vom Code
  numberstyle=\tiny, % kleine Zeilennummern
  numbersep=5pt,
  breaklines=true,
  showstringspaces=false,
  xleftmargin=0.7cm,
  frame=tb,
	moredelim=[is][\itshape]{__}{__},
	moredelim=[is][\textcolor{lbrown}]{~~}{~~},
	moredelim=[is][\textcolor{blue}]{++}{++},
	moredelim=[is][\itshape\bfseries\textcolor{darkblue}]{**}{**}
}
\lstdefinestyle{XML} {
    language=XML,
    captionpos=b, % Beschriftung ist unterhalb
    basicstyle=\ttfamily\small,
    columns=fullflexible,
    commentstyle=\color{gray}\upshape,
    %keywordstyle=\color{orangered},
    stringstyle=\ttfamily\color{black}\normalfont,
    tagstyle=\color{darkblue}\bfseries,
    extendedchars=true, 
    breaklines=true,
    breakatwhitespace=true,
    emph={},
    emphstyle=\color{red},
    morestring=[b]",
    morecomment=[s]{<?}{?>},
    morecomment=[s][\color{forestgreen}]{<!--}{-->},
    morekeywords={attribute,xmlns,version,type,release},
    frame=tb
}
\lstdefinestyle{XML_LN} {
    language=XML,
    captionpos=b, % Beschriftung ist unterhalb
    basicstyle=\ttfamily\small,
    columns=fullflexible,
    commentstyle=\color{gray}\upshape,
    %keywordstyle=\color{orangered},
    stringstyle=\ttfamily\color{black}\normalfont,
    tagstyle=\color{darkblue}\bfseries,
    numbers=left, % Zeilennummern links vom Code
  	numberstyle=\tiny, % kleine Zeilennummern
  	numbersep=5pt,
    extendedchars=true, 
    breaklines=true,
    breakatwhitespace=true,
    emph={},
    emphstyle=\color{red},
    morestring=[b]",
    morecomment=[s]{<?}{?>},
    morecomment=[s][\color{forestgreen}]{<!--}{-->},
    morekeywords={attribute,xmlns,version,type,release},
    frame=tb
}
\lstdefinestyle{JSON} {
    captionpos=b, % Beschriftung ist unterhalb
    basicstyle=\small,
    columns=fullflexible,
    extendedchars=true, 
    breaklines=true,
    breakatwhitespace=true,
    string=[s]{"}{"},
    stringstyle=\color{darkblue}\ttfamily\bfseries,
    comment=[l]{:},
    commentstyle=\color{black}\ttfamily,
		frame=tb
}
\usepackage[simplified]{pgf-umlcd}
\renewcommand{\umltextcolor}{black}
\renewcommand{\umlfillcolor}{white}
\renewcommand{\umldrawcolor}{black}
\setcounter{secnumdepth}{2}
\setcounter{tocdepth}{2}
\setlength{\parskip}{\medskipamount}
\setlength{\parindent}{0pt}
\usepackage{soul}
\makeindex
\usepackage{nomencl}
\providecommand{\printnomenclature}{\printglossary}
\providecommand{\makenomenclature}{\makeglossary}
\makenomenclature
\AfterFile{t1lmss.fd}{
  \DeclareFontShape{T1}{lmss}{b}{n}
  {<->ssub*lmss/bx/n}{}
}
\usepackage{textcomp}
% Fehlerhaft bei mir. Teilweise werden Buchstaben in der PDF View in Eclipse
% nicht angezeigt.
%\usepackage[charter]{mathdesign}
\usepackage{footmisc}
\DeclareSymbolFont{cmsy}{OMS}{cmsy}{m}{n}
\DeclareSymbolFontAlphabet{\mathcal}{cmsy}
\usepackage{fancyhdr}
\usepackage{setspace}
\PassOptionsToPackage{hyphens}{url}\usepackage{hyperref}
\onehalfspacing
\pagestyle{headings}
\newcommand*{\ORIGchapterheadendvskip}{}
\let\ORIGchapterheadendvskip=\chapterheadendvskip
\renewcommand*{\chapterheadendvskip}{
    {
        \setlength{\parskip}{0pt}
        \noindent\rule{.3\textwidth}{3pt}\rule[2.5pt]{.7\linewidth}{.5pt}\par
    }
    \vskip1.2em
}
\makeatletter
\providecommand{\toclevel@lstlisting}{0}
\makeatother
\addtokomafont{caption}{\small}
\setkomafont{captionlabel}{\sffamily\bfseries}
\setcapindent{1em}
\renewcommand \theparagraph {(\arabic{paragraph})}
\pagestyle{fancy}
\usepackage{calc}
\renewcommand{\chaptermark}[1]{\markboth{#1}{}}
\renewcommand{\sectionmark}[1]{\markright{\thesection\ #1}}
\fancyhf{}
\fancyfoot[RO]{\rightmark\quad{\bfseries\thepage}}
\fancyfoot[LE]{{\bfseries\thepage}\quad\leftmark}
\renewcommand{\headrulewidth}{0pt}
\fancypagestyle{plain}{
  \fancyhf{}
  \fancyfoot[RO]{{\bfseries\thepage}}
  \fancyfoot[LE]{{\bfseries\thepage}}
  \renewcommand{\headrulewidth}{0pt}
}
\usepackage{booktabs}
\usepackage{pifont}
\setlength{\nomlabelwidth}{.20\hsize}
\renewcommand{\nomlabel}[1]{#1 \dotfill}
\hyphenation{name-space}
\usepackage{array}
\usepackage{bigstrut}
\graphicspath{{../img/}}
\newcolumntype{Y}{>{\centering\arraybackslash}X}
\usepackage{bibgerm}
\usepackage{graphicx}

\makeatother

\begin{document}
\pagenumbering{gobble}\begin{titlepage}
\newcommand{\Titel}[2][\textwidth]{\renewcommand{\baselinestretch}{1.0}\hfil
            \parbox{#1}{\huge\bfseries\centering #2}\hfil\par}
\renewcommand{\subtitle}[2][\textwidth]{\renewcommand{\baselinestretch}{1.0}\hfil
            \parbox{#1}{\LARGE\centering\mbox{}\llap{--~}#2\rlap{~--}}\hfil\par}
\newcommand{\addLine}[2][\textwidth]{\renewcommand{\baselinestretch}{1.0}\hfil
            \parbox{#1}{\large\centering #2}\hfil\par}


\begin{textblock}{40}(30,27)
 \includegraphics[scale=0.7]{img/unilogo.pdf}
\end{textblock}

\begin{textblock}{40}(120,15)
 \includegraphics[scale=0.07]{img/vs-color.pdf}
\end{textblock}

\addLine{}
\vspace{0.2\textheight}

\Titel{Automatisiertes Aufsetzen eines Kubernetes-Clusters auf Raspberry Pis mithilfe von Ansible-Playbooks}
%\vspace{0.09\textheight}

\newcommand*{\student}{KL}
\newcommand*{\matrikelnummer}{-}
\newcommand*{\betreuerA}{RH}
\newcommand*{\betreuerB}{HB}
\IfFileExists{titel-info}{\input{titel-info}}

\vspace{0.05\textheight}
\addLine{Seminararbeit von}
\addLine{\LARGE \student}
\addLine{Matrikelnummer:}
\addLine{\LARGE \matrikelnummer}
\addLine{Vorgelegt im Fachgebiet}
\addLine{\LARGE Verteilte Systeme}
\vspace{0.05\textheight} 
\addLine{
  \begin{tabbing}
   Betreuer: \quad ~\= \betreuerA\\
   Betreuer: \quad ~\= \betreuerB
  \end{tabbing}
}
\vspace{0.05\textheight} 
\addLine{\today}
\vspace{0.05\textheight} 
\addLine{Universität Kassel}
\addLine{Fachbereich Elektrotechnik und Informatik}
\addLine{Wintersemester 2019/2020}
\end{titlepage}

%\cleartooddpage[\thispagestyle{empty}]
%\newpage
%\cleartooddpage[\thispagestyle{empty}]
%\thispagestyle{empty}

%\thispagestyle{empty}
%\cleardoublepage



\cleardoublepage{}

\chapter*{Erklärung}

\vfill

Hiermit versichere ich, dass ich die vorliegende Seminararbeit selbstständig
verfasst und nur die angegebenen Hilfsmittel und Quellen benutzt habe. Alle
Stellen, die wörtlich oder sinngemäß aus veröffentlichten oder
unveröffentlichten Schriften entnommen sind, habe ich als solche kenntlich
gemacht. Diese Seminararbeit wurde bis jetzt noch nicht veröffentlicht und
wurde bisher noch in keinem anderen Prüfungsamt vorgelegt.
\newline
\newline

\qquad Kassel, den \today \hfill
\rule{0.45\textwidth}{0.4pt}

\vfill 


\cleardoublepage{}

\addtocontents{toc}{\protect\enlargethispage{2\normalbaselineskip}}

\setcounter{tocdepth}{1}
\tableofcontents{}

\listoffigures

\thispagestyle{empty}
%\cleartooddpage[\thispagestyle{empty}]
\pagenumbering{arabic}
\setcounter{page}{1}

%
\chapter{Einleitung}

\cite{Misc1,book1}

\section{Zielsetzung}

\section{Inhaltlicher Aufbau der Arbeit}

%\include{Verwandtearbeiten}\include{Grundlagen}
%\include{Entwurf}
%\include{Implementierung}
%\include{Evaluierung}
%\include{Fazit}

\chapter{Einleitung}\label{ch:einleitung}

Kubernetes ist eine moderne Technologie, die skalierbare Applikationen ermöglicht.
Sie beruht auf dem Prinzip, mehrere Rechner miteinander zu vernetzen und ihre gesamten Ressourcen effizient zu nutzen.
In Produktivumgebungen werden dazu leistungsstarke Maschinen oder Cloud-Instanzen eingesetzt.
% TODO finanz nicht so wichtig, vorteile von lokal hervorheben, warum überhaupt cluster? fork computing, mobile edge computing
Im Entwicklungsbetrieb kann es jedoch von Vorteil sein, aus finanziellen Gründen auf die Leistungsfähigkeit und die gute Anbindung eines Rechenzentrums zu verzichten und stattdessen auf lokal betriebene Systeme zu setzen.
% TODO raspi verlinken
Eine besonders günstige Option stellen hierbei Einplatinencomputer wie der Raspberry Pi dar.

Es sind viele Schritte nötig, um einen Kubernetes-Cluster einzurichten und je mehr Nodes eingerichtet werden sollen, umso häufiger müssen die immer gleichen Schritte durchgeführt werden.
Mithilfe des Automatisierungs-Werkzeugs Ansible können diese Aufwände automatisiert und somit vereinfacht und beschleunigt werden.
Nachdem mit wenigen Handgriffen das Standardbetriebssystem Raspbian installiert wurde, werden alle weiteren Schritte von Ansible-Playbooks automatisch erledigt.

In dieser Seminararbeit werden zunächst die verwendeten Technologien, Kubernetes und Ansible, kurz vorgestellt (Kapitel~\ref{ch:technologien}).
Anschließend wird die Vorgehensweise zum Aufsetzen eines Clusters mithilfe der Playbooks übersichtlich zusammengefasst (Kapitel~\ref{ch:anwendung}).
Danach erfolgt eine ausführliche Erläuterung der Funktionsweise der Playbooks (Kapitel~\ref{ch:umsetzung}).
Zuletzt werden mögliche Alternativen zu den eingesetzten Technologien vorgestellt (Kapitel~\ref{ch:alternativen}) und ein Ausblick auf mögliche Weiterentwicklungen gegeben (Kapitel~\ref{ch:zusammenfassung}).
\chapter{Technologien}\label{ch:technologien}

Wir arbeiten mit coolen Technologien

\section{Ansible}\label{sec:ansible}

Ansible macht Sachen automatisch

\section{Docker}\label{sec:docker}

Docker ist das mit dem Wal

\section{Kubernetes}\label{sec:kubernetes}

Kubernetes schubst Container

\chapter{Anwendung}\label{ch:anwendung}

Um mithilfe der Playbooks aus diesem Projekt einen Kubernetes-Cluster einzurichten, müssen zunächst die WiFi-Infrastruktur und die Raspberry Pis vorbereitet werden.
Dazu werden die SD-Karten einzeln mit Raspbian beschrieben und mit dem Playbook \texttt{local-""raspbian.yaml} für den Headless-Betrieb\footnote{Betrieb ohne Bildschirm} konfiguriert, mit Strom versorgt und gestartet.
Sobald alle Raspberry Pis online sind, wird Kubernetes mit dem Playbook \texttt{kubernetes.yaml} aufgesetzt.

\begin{figure}[h]
    % https://www.draw.io/?title=ansible-kubernetes-anwendung#R7Vlbd6IwEP41PuoBgoqPK73stt2trdt2%2B7QnSoTUQNgQqu6v3wBBwFhLd721Z33wkEkml2%2Fmm5lAA9j%2B%2FJzB0PtKHUQahubMG%2BCkYRidblf8J4JFJgAdLRO4DDuZSC8EQ%2FwbSWE%2BLMYOiioDOaWE47AqHNMgQGNekUHG6Kw6bEJJddUQukgRDMeQqNIH7HAvk1pGt5B%2FRtj18pX1Ti%2Fr8WE%2BWJ4k8qBDZyUROG0Am1HKsyd%2FbiOSYJfj8vBl8UCupp3zi5voF7zrX37%2Fdt%2FMJjt7i8ryCAwF%2FK%2Bnvn6M4N2nm%2Blvt%2Bs%2F65bZu7%2F50ZRnfYYklnjJs%2FJFDiByBJ6ySRn3qEsDSE4LaZ%2FROHBQsowmWsWYK0pDIdSF8AlxvpDOAWNOhcjjPpG9Nc8ncYhozMZow6GAdDPIXMQ3jLOycckBS74i0TtH1EecLcQAhgjk%2BLnqUFD6pbscV2AvHiT8bzAFUExxC6NwhGGQaAYRh4RgJCBSLFTgn4A58zBHwxCmEM0Ep%2F8F62fEOJpvRCfvNSVNZJzoyeasIJ2exwSvRLhcbft4gkO4Mppj%2FiNRb7Vl67HUczKXM6eNRd4IxHlLSknzsdxXqKWtXG%2FLtLFq0gYYR8UbS%2BHNgMDFiNJpw%2BgQcZD%2BiIknl6eIZZIJFYiJZAIlIJ1fcRLF%2B7bASHBMdH1Ds0KcqxMq8kqTLWmZTSZ2nc1XXaNYFcbRpGGDRt%2F23g199XZN%2FrZ3xV%2B995%2B%2Fb%2BKvXjfvgaPir64mvuFJ8xIKz00Tn%2Fi7wkE8bw7sZNM4IMh9LywCe0yC5On69IthfO1fxfZZ%2FMSjW%2FehaX7I%2Bs5QHX18ez%2B%2BnNoXZ5%2FC7sQ2m2LvI1kQH4uj59ve4OhpxYdYuqkBfnfuvrwqHSxpgI6CcYCwCt8%2BM0kpjxRZ5ZVMolfySJFW9pNJgFkzk%2BjHlUqAWgs%2BwUPaXv8b22vvw%2Fage1y2V1PdEAdOshQhKI2nUfI6ComLtIhv4mRnalD1qD%2BKo9cDKiTYDcTzWGAurgu7i7BAWynLawZYa1cFhZrDPkJBsXVXlqoDitOLprRm29JamviZJjBBx8qj5yJPii3NsNpd3cz%2BV6bPCClnXLHicov%2FwJ%2BuYtmd3aOnsShyAsRR9KHu0GZ7pRxq768aWktWXTHp8ZP1ADXHerLqRtWaq1bKoso2%2BLjWduqt%2BPhtVw60my5kr9cW5qFqi03bXn01X76oZVXFkQam1SqidmTaxnuJIbsdoMueb7oXJ95dYI1Ddr0GUdW7CcFhhKoIRR4Mk34UOEMOedI7wYTYlFCWqgEt%2FSVDOaNTtK7nBfjqIv8izG1r5e2PijJYAzLYFcg1wn8GUv5h1FgPtj93k0%2FDrQmhs7En%2FLyVevtP4wWP3hW%2BvX3BK5rFZ90soBffxsHpHw%3D%3D
    \centering
    \includegraphics[width=\textwidth]{img/anwendung.pdf}
    \caption{Ablaufdiagramm zur Einrichtung eines Clusters}
\end{figure}

\section{Vorbereitung}\label{sec:vorbereitung}
Zum erfolgreichen Ausführen dieser Anleitung müssen folgende Voraussetzungen erfüllt sein:

\textbf{Raspberry Pis} in beliebiger Anzahl und ebenso viele Speicherkarten und Netzteile stehen bereit.

\textbf{Ein Linux-Rechner} steht bereit, um die Speicherkarten zu beschreiben und die Ansible-Playbooks auszuführen. Dafür sind Balena Etcher\footnote{\url{https://www.balena.io/etcher/} -- \today} und Ansible\footnote{\url{https://www.ansible.com/} -- \today} installiert, das Git-Repository zu diesem Projekt mit den Verzeichnissen \texttt{inventory} und \texttt{playbook} ist lokal verfügbar und auf dem aktuellen Stand und ein Image von Raspbian Lite\footnote{\url{https://downloads.raspberrypi.org} -- \today} ist heruntergeladen.
% https://downloads.raspberrypi.org/raspbian_lite/images/raspbian_lite-2020-02-14/2020-02-13-raspbian-buster-lite.zip

\textbf{Ein Terminal} mit \texttt{playbook} als Arbeitsverzeichnis ist geöffnet.

\textbf{Ein WiFi-Access Point} mit Internetzugriff ist in Betrieb. Seine Einstellungen (IP, SSID, WPA2-Key) entsprechen den Angaben in den Dateien \texttt{inventory/""group\_vars/""all.yaml} und \texttt{playbook/""local-""raspbian.yaml}. Der zuvor erwähnte Linux-Rechner ist mit dem Access Point verbunden.

\textbf{Die Inventory-Datei} \texttt{inventory/""k8s-cluster.yaml} bildet den derzeitigen Cluster ab -- enthält also keine Einträge unter \texttt{hosts}, falls ein neuer Cluster eingerichtet werden soll:

\begin{lstlisting}[language=yaml, caption=Leere Inventory-Datei]
nodes:
  hosts:
\end{lstlisting}

\textbf{Ein SSH-Schlüsselpaar} ist generiert. Der private Schlüssel ist auf dem Linux-PC unter \texttt{\$HOME/"".ssh} hinterlegt und der öffentliche Schlüssel im Playbook \texttt{local-""raspbian"".yaml} in dem Array \texttt{sshKeys}.

\section{Raspbian installieren und Cluster einrichten}\label{sec:raspbian-installieren}
Die folgenden Schritte müssen für jeden Raspberry Pi einzeln durchgeführt werden.

\begin{enumerate}
    \item Speicherkarte in den Linux-Rechner einlegen.
    \item Raspbian-Image mit Balena Etcher auf der Speicherkarte installieren.
    \item Raspbian-Playbook ausführen:
        \begin{lstlisting}
sudo ansible-playbook -i ../inventory/k8s-cluster.yaml local-raspbian.yaml
\end{lstlisting}
    \item Speicherkarte in den Raspberry Pi einsetzen und starten.
\end{enumerate}

Wenn alle Raspberry Pis gestartet sind, wird die Installation mit dem Kubernetes-Playbook fortgesetzt:

\begin{lstlisting}
ansible-playbook -i ../inventory/k8s-cluster.yaml kubernetes.yaml
\end{lstlisting}
\caption{CLI-Befehl zur Ausführung des Raspbian-Playbooks}

Der Kubernetes-Cluster ist anschließend einsatzbereit.

\section{Weitere Nodes hinzufügen}\label{sec:weitere-nodes-hinzufügen}

Sollen zu einem fertigen Cluster weitere Nodes hinzugefügt werden, kann auch dafür diese Anleitung ab Abschnitt~\ref{sec:vorbereitung} verwendet werden.
Die Inventory-Datei darf dann nicht leer sein, sondern muss die bereits vorhandenen Nodes enthalten.

\chapter{Umsetzung}
Schöne Flowcharts!
\chapter{Alternativen}\label{ch:alternativen}
\section{Alternativen zu Ansible}\label{sec:ansible-alternativen}

Ansible hat mehrere Konkurrenzprodukte, zum Beispiel Puppet und Chef.

Alle sind etabliert und eignen sich zur Automatisierung von Konfigurationen über Netzwerkverbindungen.
Ein wesentlicher Unterschied besteht in der grundlegenden Funktionsweise.
Während Ansible genau dann arbeitet, wenn ein Playbook auf dem Steuerungsrechner ausgeführt wird (Push-Prinzip), setzen Puppet und Chef Agenten-Software auf den Remote-Hosts voraus, die selbst den Zeitpunkt der Konfiguration bestimmen Pull-Prinzip).
Dafür muss dauerhaft ein Puppet- beziehunsgsweise Chef-Server bereitstehen, um zum gegebenen Zeitpunkt die nötigen Konfigurationen bereitzustellen.
Puppet und Chef eignen sich dadurch eher für Anwendungsfälle, in denen Zeit keine wichtige Rolle spielt oder nicht alle Remote-Hosts gleichzeitig verfügbar sind, zum Beispiel Büro-Rechner, die für begrenzte Dauer und zu unterschiedlichen Zeiten verwendet werden.

In diesem Projekt laufen alle Remote-Hosts gleichzeitig und eine unverzügliche Ausführung der Konfiguration auf Knopfdruck ist erwünscht.
Das macht Ansible zu einer guten Wahl.

\section{Alternativen zu Kubernetes}\label{sec:kubernetes-alternativen}
Docker Swarm?

Fleet?

Ubuntu?
\chapter{Zusammenfassung}\label{ch:zusammenfassung}

\section{Ausblick}\label{sec:ausblick}

Image flashen per Ansible

Idempotenz (kubeadm init)

Globale Variablen (kubeadm join)

\section{Fazit}\label{sec:fazit}

Alles cool

Die benötigte Zeit für die gesamte Einrichtung beträgt ca. 10 Minuten pro Raspberry Pi plus etwa 30 Minuten für den gesamten Cluster.



\bibliographystyle{gerplain}
\bibliography{references}

\end{document}
